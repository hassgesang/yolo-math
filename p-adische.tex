\documentclass[a4paper,fleqn]{article}
\usepackage{amsmath, amsfonts, amsthm, amssymb,}

\usepackage{multicol}
\usepackage{geometry} \newgeometry{tmargin=2.5cm, bmargin=2.5cm, lmargin=2cm, rmargin=2cm} 

\begingroup\makeatletter\@for\theoremstyle:=definition,remark,plain\do{\expandafter\g@addto@macro\csname th@\theoremstyle\endcsname{\addtolength\thm@preskip\parskip}}\endgroup

\usepackage{alegreya}
\usepackage[plmath]{polski}

%\theoremstyle{definition}
\newtheorem{prbh}{Zadanie}
\newtheorem{prb}[prbh]{\'Cwiczenie}
\newtheorem{fkt}{Fakt}
\newtheorem{sol}{Problem}
\newtheorem{dff}{Definicja}

\usepackage[parfill]{parskip}
\usepackage[none]{hyphenat}
\usepackage{microtype}

\usepackage[polish]{babel}
\usepackage[T1]{fontenc}
\usepackage[utf8]{inputenc}
\selectlanguage{polish}

\newenvironment{enumx}{\begin{enumerate}
	\setlength{\itemsep}{0pt}
	\setlength{\parskip}{0pt}
	\setlength{\parsep}{0pt}}
{\end{enumerate}}

\newenvironment{itemx}{\begin{itemize}
	\setlength{\itemsep}{0pt}
	\setlength{\parskip}{0pt}
	\setlength{\parsep}{0pt}}
{\end{itemize}}

\title{\textbf{Liczby $p$-adyczne}}
\author{}
\date{}

\begin{document}
%\maketitle
\begin{multicols}{2}
\begin{center}\textbf{\huge Liczby $p$-adyczne}\end{center}

Celem tych notatek jest pokazanie tego, co daje $p$-adyczna ścieżka rozwoju. Naszym zamiarem nie jest naukowa ekspedycja, ale zwiedzanie, chcemy tylko zapoznać się z dziwnym światem liczb $p$-adycznych i jego prawami. Rozpoczynamy od analogii Hensela, potem badamy własności wartości bezwzględnych. Definiujemy $p$-adyczną waluację i normę, by móc udowodnić tw. Ostrowskiego i rozszerzyć ciało $\mathbb Q$ w alternatywny wobec konstrukcji ciągów Cauchy'ego sposób, co stanowi zalążek dla analizy $p$-adycznej. Drogi Czytelniku, mam nadzieję, że znajdziesz przyjemność w obcowaniu z tą piękną córką matematuni!

\textbf{Żarcik}. \emph{Do $p$-adycznego baru wchodzi nieskończona liczba matematyków. Piewszy zamawia $p-1$ piw, drugi $(p-1)^2$, trzeci $(p-1)^3$, i tak dalej. Barman, młody student analizy $p$-adycznej, wystawia jeden kufel, a matematycy jednogłośnie stwierdzają, że są idiotami i nalewają piwo do pełna.}

%\textbf{Zastosowania liczb $p$-adycznych, czyli do czego to się przydaje}: dowód faktu, że jedyna l. całkowita w  ciągu sum częściowych szeregu $\sum_n 1/n$ to $1$, atakowanie pewnych równań diofantycznych,  {dowód twierdzenia Monsky'ego (niemożliwe jest rozcięcie kwadratu na nieparzyście wiele trójkątów o tym samym polu)}, mechanika kwantowa (w świecie liczb $p$-adycznych można uniknąć renormalizacji, tj. traktowania rozbieżnych szeregów jak skończonych wielkości).

\section{Motywacja}
Liczby $p$-adyczne zostały wprowadzone do matematyki przez Hensela\footnote{\textsc{Kurt Hensel} (29.12.1861 -- 01.06.1941), matematyk niemiecki}.
Wydaje się, że jego główną motywacją była analogia pomiędzy pierścieniem liczb całkowitych $\mathbb Z$ (razem z ciałem ułamków $\mathbb Q$) i pierścieniem wielomianów o zespolonych współczynnikach $C[X]$ (razem z ciałem funkcji wymiernych $\mathbb C(X)$). Mówiąc dokładniej: $f(X) \in \mathbb C(X)$ jest funkcją wymierną, jeżeli jest ilorazem dwóch wielomianów: \[f(X) = \frac{P(X)}{R(X)}, \text{ gdzie } P(X)\in \mathbb C(X), R(X)\in \mathbb C(X) \text { i } R(X) \neq 0.\]
W podobny sposób możemy przedstawić dowolną liczbę wymierną $x \in \mathbb Q$ jako iloraz dwóch liczb całkowitych: \[x = \frac{a}{b}, \text{ gdzie } a\in \mathbb Z, b \in \mathbb Z \text { i } b \neq 0.\]
Co ciekawe, oba pierścienie są do siebie podobne: zarówno $\mathbb Z$ jak i $\mathbb C[X]$ są pierścieniami z jednoznacznością rozkładu (\emph{UFD, unique factorization domain}). Każda liczba całkowita może zostać jednoznacznie przedstawiona jako iloczyn $\pm 1$ i skończonej liczby liczb pierwszych, każdy wielomian może zostać jednoznacznie zapisany jako $P(X) = a(X - z_1)(X - z_2)\cdot\ldots\cdot(X - z_n)$, gdzie $a$ i $z_1, z_2, \cdots, z_n$ są liczbami zespolonymi. Hensel zauważył, że \emph{liczby pierwsze $p\in\mathbb Z$ są odpowiednikiem wielomianów pierwszego stopnia $X - z \in \mathbb C[X]$}. Dotychczas nie powiedzieliśmy niczego godnego uwagi. Henselowi udało się jednak dostrzec, że analogia jest głębsza. Mając dany wielomian $P(X)$ i liczbę $z \in \mathbb C$ możemy zapisać (np. korzystając z rozwinięcia w szereg Taylora\footnote{\textsc{Brook Taylor} (18.08.1685 -- 29.12.1731), matematyk angielski.}) wielomian w formie
\begin{align*}
P(X) & = a_0 + a_1(X-z) + a_2(X-z)^2 + \cdots + a_n(X-z)^n \\
& = \sum_{k=0}^n a_k(X-z)^k, \text{ gdzie } a_k \in\mathbb C
\end{align*}
To działa też dla liczb całkowitych (na początek tylko dodatnich): mając liczbę całkowitą $m$ i liczbę pierwszą $p$ możemy zapisać $m$ w systemie pozycyjnym o podstawie $p$:
\begin{align*}
m & = a_0 + a_1p+ a_2p^2 + \cdots + a_np^n \\
&  = \sum_{k=0}^n a_kp^k, \text{ gdzie } a_k \in\mathbb Z \text{ i } 0 \le a_k \le p-1
\end{align*}

Dlaczego takie rozwinięcia są ciekawe? Dają nam pewną lokalną wiedzę! Rozwinięcie w potęgi $(X-z)$ daje nam informację o tym, czy $P(X)$ znika w $z$ i do którego rzędu. Rowinięcie o podstawie $p$ mówi nam, czy i jak bardzo $m$ jest podzielne przez $p$. Oto przykład: $72 = 0 + 0 \cdot 3^1 + 2\cdot 3^2+ 2\cdot 3^3$, czyli $3^2\mid 72$.

Dla wielomianów i ich ilorazów możemy posunąć się jeszcze dalej. Weźmy $f(X) \in \mathbb C(X)$ i $z\in\mathbb C$, zawsze istnieje rozwinięcie
\begin{align*}
f(X) & = \frac{P(X)}{R(X)} = \sum_{k\ge n_0} a_k(X-z)^k \\
& = a_{n_0} (X-z)^{n_0} + a_{1+n_0} (X-z)^{1+n_0} + \dots
\end{align*}

To nic innego jak tylko rozwinięcie w szereg Laurenta\footnote{\textsc{Pierre Alphonse Laurent} (18.07.1813 -- 02.09.1854), matematyk francuski.}, znane z analizy zespolonej, które różni się znacząco od poprzedniego (szeregu)! Po pierwsze, być może $n_0<0$, tj. pierwszy wykładnik jest ujemny, czyli $z$ jest pierwiastkiem $R(X)$, ale nie $P(X)$. W języku analizy mówimy, że $f(X)$ ma biegun w $z$ rzędu $-n_0$. Po drugie, szereg zazwyczaj nie jest skończony. Okazuje się, że jest on skończony wtedy i tylko wtedy, gdy zapiszemy funkcję $f(X) = P(X)/R(X)$ jako nieskracalny ułamek, a $R(X)$ okaże się być potęgą $(X-z)$. Okazuje się też, że $f(z')$ jest zbieżnym szeregiem, o ile $z'$ jest prawie (ale nie dokładnie!) równe $z$.

Traktujemy szeregi jako formalne obiekty i nie przejmujemy się tym, czy są zbieżne. Oto przykład.
\[f(X) = \frac{X}{X-1}\]
Będziemy rozwijać tę funkcję dla różnych $z$, co jest po prostu ćwiczeniem z analizy. Jeśli $z=0$, to 
\begin{align*}
\frac{X}{X-1} & = -X \cdot \frac{1}{1-X} = -X (1+X+X^2+X^3+\dots) \\
& = -X -X^2-X^3-X^4-\dots
\end{align*}
pokazuje nam, że $f(0) = 0$ z krotnością jeden. Weźmy teraz $z=1$ i zauważmy, że \[\frac{X}{X-1} = \frac{1+X-1}{X-1} = (X-1)^{-1}+ 1\]
To oznacza, że w tym miejscu mamy do czynienia z biegunem rzędu jeden (a  rozwinięcie jest skończone!). Ustalmy wreszcie $z=2$, nie jest to ani biegun, ani miejsce zerowe, mamy \[\frac{X}{X-1} = 2 - (X-2) + (X-2)^2-(X-2)^3+\dots\]
Czas na puentę. Każda funkcja wymierna może być wyrażona jako szereg, którego wyrazy są krotnościami potęg \emph{pierwszego}\footnote{Informacja dla wiecznie nienasyconych: ideały generowane przez elementy postaci $(X-z)$ są dokładnie ideałami pierwszymi pierścienia $\mathbb C[X]$, więc mamy prawo nazywać je pierwszymi} wielomianu $(X-z)$. Pomysłem Hensela było rozciągnięcie analogii między $\mathbb Z$ i $\mathbb C[X]$ na takie rozwinięcia. Radzimy sobie dobrze z liczbami całkowitymi, co z ułamkami? Zgodnie z intuicją, zapisujemy licznik i mianownik jako sumę potęg $p$, a następnie dzielimy formalnie. Najlepiej widać to na przykładzie, ustalmy więc $p=3$, $a=24$, $b=17$. Łatwo widać, że:
\begin{align*}
a & = 24= 0+2\cdot+2\cdot 3^2 = 2p+2p^2\\
b & = 17 = 2 + 2 \cdot 3 + 1 \cdot 3^2 = 2+2p+2p^2 \\
\frac{a}{b} & = \frac{2p+2p^2}{2+2p+p^2} = p+p^3+2p^5+p^7+p^8+2p^9+\dots
\end{align*}



\textbf{Tutaj czegoś brakuje.}

%Rozwińmy funkcję $\log (x+1)$ w szereg MacLaurina: mamy $\log(x+1) = x - x^2/2 + x^3/3 - x^4/4 + \dots$. Wiemy, że potęgi dwójki są małe w $\mathbb Q_2$, podstawmy $x= -2$ do rozwinięcia, by policzyć $\log -1$.
%\[\log (-1) = \log(1-2) = -\left(2+\frac{2^2}{2}+\frac{2^3}{3}+\frac{2^4}{4} + \cdots\right)\]
%Szereg jest bardzo rozbieżny w $\mathbb R$, ale w magicznym świecie liczb $2$-adycznych jest jak najbardziej sumowalny. A skoro jest sumowalny, to musi zbiegać do zera, albowiem $2\log(-1) = \log(1) = 0.$ To oznacza, że sumy częściowe 
\section{Wstęp}
%\begin{abstract}

%



W tym rozdziale $\mathbb K$ będzie oznaczało dowolne ustalone ciało, zaś $\mathbb R_+$ będzie zbiorem $\{x\in\mathbb R:x\ge 0\}$.  Przypomnijmy, że ciało to przemienny pierścień, w którym wszystkie niezerowe elementy są odwracalne, tzn. jest to taka struktura $(K,\oplus, \odot,1,0)$, która spełnia określone warunki:
\begin{enumx}
\item zbiór $K$ zawiera co najmniej dwa elementy, $0$ oraz $1$,
\item $\oplus$ i $\odot$ są działaniami wewnętrznymi w $K$ i dla dowolnych $a,b,c\in K$ prawdą jest, że:
\begin{itemx}
\item $a  \oplus  (b  \oplus  c) = (a  \oplus  b)  \oplus  c$,
\item $a \oplus 0=a$,
\item $(\exists d \in K)(a \oplus d= 0)$,
\item $a \oplus b=b \oplus a$,
\item $a \odot (b \odot c) = (a \odot b) \odot c$,
\item $a \odot 1 = a$,
\item $a \odot (b \oplus c)= (a \odot b)  \oplus  (a \odot c)$,
\item $a \odot b = b \odot a$,
\end{itemx}
\item każdy niezerowy element jest odwracalny, co oznacza, że
\begin{itemx}
\item $(\forall a \in K\setminus\{0\})(\exists b\in K)(a\odot b = 1)$.
\end{itemx}
\end{enumx}

Zacznijmy od możliwie najprostszej definicji wartości bezwzględnej na $\mathbb K$. Postaramy się uchwycić najważniejsze własności klasycznej wartości bezwzględnej.

\begin{dff} Wartość bezwzględna (norma) na $\mathbb K$ to dowolna funkcja $|\cdot| : \mathbb K \rightarrow \mathbb R_+$, która jest:
\begin{itemx}
\item dodatnio określona: $|x| = 0 \iff x = 0$ 
\item multiplikatywna: $(\forall x, y\in\mathbb K)(|xy|=|x|\cdot|y|)$
\item podaddytywna: $(\forall x, y\in\mathbb K)(|x+y|\le|x|+|y|)$
\end{itemx}
Mówimy, że wartość bezwzględna jest niearchimedesowska, jeżeli spełnia dodatkowy warunek:
\begin{itemx}
\item $(\forall x,y\in\mathbb K)(|x+y| \le \max(|x|,|y|))$ 
\end{itemx}
\end{dff}

Zauważmy, że czwarty warunek implikuje trzeci, ponieważ $\max(|x|,|y|)$ jest z pewnością mniejsze od sumy $|x| + |y|$. Ważność tych norm i geneza ich nazwy zostanie przedyskutowana później, na razie przyjmijmy do wiadomości, że są dość popularne. Czas na przykłady!

\textbf{Klasyczna wartość bezwzględna}: niech $\mathbb K = \mathbb Q$ i ustalmy, że dla $x\in\mathbb Q$ mamy $|x| = \max(x,-x)$. Jest ona archimedesowska, ponieważ czwarty warunek z definicji nie zachodzi dla $x = y = 1$. Zazwyczaj określana jako \emph{wartość bezwzględna w nieskończoności}, jest oznaczana przez $|\cdot|_\infty$.

\textbf{Trywialna wartość bezwzględna}: przyjmijmy, że $|x| = 1$, jeśli $x$ nie jest zerem i $|0| = 0$. W ten sposób zdefiniowaliśmy najprostszą niearchimedesowską wartość bezwzględną dla dowolnego $\mathbb K$, którą będziemy często pomijać w twierdzeniach.

W oparciu o samą definicję możemy już coś wywnioskować. Warto na przykład zauważyć, że dla skończonych ciał nasza teoria trywializuje się! 

\begin{fkt}Trywialna norma jest jedyną, jaką możemy wprowadzić na skończonym ciele $\mathbb K$. \end{fkt}
\begin{proof}Posłużymy się \textbf{twierdzeniem Lagrange'a}: w grupie skończonej $G$ rząd dowolnej podgrupy jest dzielnikiem rzędu $G$. Niech $|\cdot|$ będzie normą na $\mathbb K$, wtedy z definicji $|0| = 0$. Równość $1 = 1 \cdot 1$ wymusza $|1| = |1| \cdot |1|$. Jedynym dodatnim rozwiązaniem jest $|1| = 1$. $\mathbb K$ jest skończonym ciałem o mocy $q$, więc dla dowolnego niezerowego $x\in\mathbb K$ (na mocy tw. Lagrange'a) mamy $|x|^{q-1} = 1$. To sprawia, że $|x| = 1$, zaś norma jest trywialna. \end{proof}
 
 Zaraz zdefiniujemy coś, co będzie obiektem naszego zainteresowania przez prawie cały czas. Niech $\mathbb K$ będzie zbiorem liczb wymiernych $\mathbb Q$, zaś $p\in\mathbb Z$ dowolną liczbą pierwszą. Chcielibyśmy wiedzieć, jak bardzo $n\in\mathbb Z$ jest podzielna przez $p$. Formalnie:

\begin{dff}Niech $p$ będzie liczbą pierwszą. $p$-adyczna waluacja na $\mathbb Z$ to funkcja $v_p:\mathbb Z\setminus\{0\} \rightarrow \mathbb R$ zdefiniowana tak: dla $n\in\mathbb Z\setminus\{0\}$ niech $v_p(n)$ będzie jedyną dodatnią liczbą całkowitą spełniającą \[n = p^{v_p(n)} \cdot n_0, \quad \text {przy czym} \quad p\nmid n_0.\] Rozszerzamy $v_p$ na ciało liczb wymiernych: jeśli $x = a/b \in\mathbb Q$, to $v_p(x) = v_p(a) - v_p(b)$.\end{dff}

Zwyczajowo przyjmujemy $v_p(0) = +\infty$, co możemy uzasadnić następująco: gdy podzielimy $0$ przez $p$, to dostaniemy $0$, które możemy znów podzielić przez $p$ -- proces powtarza się dowolnie wiele razy.

\begin{fkt}Jeśli $x\in\mathbb Q$, to wartość $v_p(x)$ nie zależy od wyboru reprezentanta $x$, ilorazu dwóch liczb całkowitych.\end{fkt}

\begin{proof}Jeśli $a/b = c/d$, to $ad = bc$, a zatem $v_p(ad) = v_p(bc)$. Z ZTA wiemy, że $v_p(ad) = v_p(a) + v_p(d)$ i $v_p(bc)= v_p(b) + v_p(c)$, czyli $v_p(a) + v_p(d)  = v_p(b) + v_p(c)$. Teraz łatwo już widać, że \[v_p\left(\frac{a}{b}\right) = v_p(a)-v_p(b) = v_p(c) - v_p(d) = v_p\left(\frac{c}{d}\right)\qedhere\]\end{proof}

%%% VP: 5 400, 7 902, 2 621, 3 123/48, 5 180*3

\begin{fkt}
Dla dowolnych $x,y \in \mathbb Q$ waluacja $p$-adyczna spełnia dwie podstawowe zależności, mianowicie
\[v_p(xy) = v_p(x) + v_p(y), \\ v_p(x+y) \ge \min(v_p(x), v_p(y)).\]

\end{fkt}
\begin{proof}Rozpatrzmy najpierw przypadek, gdy $x,y$ są liczbami całkowitymi. Wtedy $x = p^ax_0$ i $y=p^by_0$, gdzie $p\nmid x_0y_0$, bez straty ogólności mamy $a\le b$. Wtedy  $xy=p^{a+b}x_0y_0$ (co dowodzi pierwszego faktu), dla dowodu drugiego: $x+y=p^ax_0+p^by_0 = p^a\left(x_0+p^{b-a}y_0\right)$. Ustalmy teraz $x=a/b$ i $y=c/d$, przy czym $x,y\in\mathbb Q$.
\begin{align*}
v_p\left(\frac{ac}{bd}\right) & = v_p\left(ac\right) - v_p\left(bd\right)  = v_p\left(a\right) +v_p\left(c\right) - v_p\left(b\right) - v_p\left(d\right) = v_p\left(\frac{a}{b}\right) +v_p\left(\frac{c}{d}\right) \\
v_p\left(\frac{ad+bc}{bd}\right) & \ge \min(v_p(ad), v_p(bc)) - v_p(bd) \\ & = \min(v_p(a) + v_p(d), v_p(b) + v_p(c)) - v_p(b) -v_p(d) \notag\\
& = \min(v_p(a) - v_p(b), v_p(c) - v_p(d)) = \min(v_p(x), v_p(y)) \qedhere
\end{align*}
\end{proof}

Chcemy lepiej rozumieć, jak działa waluacja, dlatego przeliczymy kilka przykładów, zyskamy też wprawę rachunkową. 

\begin{prb}Oblicz: $v_{3}(54)$,  $v_{2}(128/7)$, $v_{3}(10^9)$, $v_3(9!)$, $v_5(400)$, $v_7(902)$, $v_2(621)$, $v_3(123/48)$, $v_5(180/3)$.\end{prb}

\begin{prb}$v_p((p^n)!) = \sum_{k=0}^{n-1}p^k$.\end{prb}

\begin{prb}Jeśli $0\le a\le p-1$, to $v_p((ap^n)!) = a\sum_{k=0}^{n-1}p^k$.\end{prb}

\begin{prb}Jeśli $x\in \mathbb Q$ i $(\forall p\in\mathbb P)(|x|_p \le 1)$, to $x\in\mathbb Z$.\end{prb}

Czas na magiczną sztuczkę. Popatrzmy na fakt, którego prawdziwość właśnie została pokazana oraz na definicję normy. Są one bardzo podobne: suma odpowiada iloczynowi (tak, jakbyśmy logarytmowali), dodatkowo nierówność zachodzi w drugą stronę. Możemy to naprawić poprzez ponowną zmianę znaku i obłożenie sumy funkcją wykładniczą. W ten sposób uzyskujemy krytyczną definicję:

\begin{dff}Dla dowolnego $x\in\mathbb Q$ definiujemy $p$-adyczną wartość bezwzględną $x$ poprzez zależność $|x|_p = p^{-v_p(x)}$, jeśli $x\neq 0$ i $|0|_p = 0$. \end{dff}

Definicja $|0|_p$ odpowiada naszej umowie, że $v_p(0) = + \infty$, jeżeli spróbujemy zinterpretować wyrażenie $p^{-\infty}$ w jedyny sensowny sposób. Aby upewnić się, że to rzeczywiście jest wartość bezwzględna, musimy sprawdzić, czy założenia definicji są spełnione. Wszystko wynika jednak z pierwszego faktu.

Spróbujmy pojąć to, co robi $p$-adyczna wartość bezwzględna. Jeżeli liczba $n$ jest bardzo podzielna przez $p$, to waluacja $v_p(n)$ będzie duża, a sama norma $|n|_p$ -- mała. Odrobina liczenia jeszcze nikomu nie zaszkodziła, a zatem…

\begin{prb}Niech $\mathbb K = \mathbb Q$, $p=7$, zaś $|\cdot| = |\cdot|_7$ będzie $7$-adyczną normą. Oblicz $|35|$, $|56/12|$, $|177553|$, $|3/686|$.\end{prb}

%%% PROBLEM 28 / GOUVEA
%%% PROBLEM 29 / GOUVEA
% Strona 35…

\begin{prb}Oblicz granicę $\lim_{n\to\infty} |p^n|_p$.\end{prb}

Wniosek: duże potęgi $p$ są małe (!), jeżeli liczymy ich $p$-adyczną wartość bezwzględną.

\begin{prb}Dla dowolnego $c\in\mathbb R$, $c>1$, równanie $|x| = c^{-v_p(x)}$ definiuje niearchimedesowską normę na $\mathbb Q$. Dlaczego upieramy się, by przyjąć $c=p$?\end{prb}

\begin{fkt}Dla każdej normy $|\cdot|$ i każdego ciała $\mathbb K$ prawdą jest, że:
\begin{enumx}
\item $|1| = 1$
\item Jeśli $x\in\mathbb K$ i $|x^n| = 1$, to $|x| = 1$
\item $|-1| = 1$
\item Dla dowolnego $x\in\mathbb K$, $|-x| = |x|$
\end{enumx}
\end{fkt}

Pokażemy teraz pierwsze poważne twierdzenie: warunek konieczny i dostateczny na to, by norma była niearchimedesowska. Zauważmy, że dla każdego ciała $\mathbb K$ mamy odwzorowanie $\mathbb Z \rightarrow \mathbb K$ (inclusion map):
\[
n \mapsto
\begin{cases}
{1 + 1 + \dots + 1 + 1} & n\text{ jedynek, jeśli } n > 0 \\
- ({1 + 1 + \dots + 1 + 1}) & -n \text{ jedynek, jeśli } n < 0 \\
   0       & \text{jeśli } n =  0 \\
  \end{cases}
\]

Dla przykładu: jeśli $\mathbb Q \subset \mathbb K$, to mamy do czynienia ze zwykłym włożeniem $\mathbb Z$ w $\mathbb Q$, jeśli $\mathbb K$ jest skończonym ciałem, to obrazem odwzorowania będzie podciało $\mathbb K$, które ma pierwszą liczbę elementów.

\begin{fkt}Niech $A\subset \mathbb K$ będzie obrazem $\mathbb Z$ w $\mathbb K$. Norma $|\cdot |$ na $\mathbb K$ jest niearchimedesowska wtedy i tylko wtedy, gdy dla każdego $a\in A$ zachodzi $|a| \le 1$. W szczególności, wartość bezwzględna na $\mathbb Q$ jest niearchimedesowska wtedy i tylko wtedy, gdy $|n|\le 1$ dla każdego $n\in \mathbb Z$.\end{fkt}

\begin{proof}Implikacja w prawo: jeśli $|\cdot|$ jest niearchimedesowska, to $|a\pm 1| \le \max(|a|, 1)$. Na mocy indukcji $(\forall a \in A)(|a| \le 1).$ Dowód twierdzenia odwrotnego wymaga małego hokus-pokus: musimy udowodnić, że dla dowolnych $x,y\in\mathbb K$ jest $|x+y|\le \max(|x|,|y|)$. Jest to oczywiste dla $y=0$. Dla $y\neq 0$ dowodzimy równoważnego warunku: $|x/y+1| \le \max (|x/y|,1)$. Niech $m\in \mathbb N$ oraz $t = x/y$, mamy wtedy
\begin{align*}|t+1|^m & = \left| \sum_{k=0}^m {m \choose k} t^k\right| \le \sum_{k=0}^m \left|{m \choose k}\right|\left| t^k\right| \le \sum_{k=0}^m \left| t^k\right| = \sum_{k=0}^m \left| t\right|^k \le (m+1) \max(1, |t|^m)\end{align*}

Pierwiastkujemy nierówność stronami: $|t+1| \le \sqrt[m]{m+1} \max(1,|t|)$ i liczymy granicę przy $m$ dążącym do nieskończoności.
\end{proof}

To tłumaczy nam, jaka jest różnica między archimedesowskimi oraz niearchimedesowskimi wartościami bezwzględnymi. \emph{Zasada Archimedesa} mówi, że mając dane $x,y\in \mathbb K$, $x\neq 0$, zawsze znajdziemy takie $n\in\mathbb N$, że $|nx| > |y|$. Jest to prawda, kiedy rozważamy klasyczną wartość bezwzględną na $\mathbb Q$ czy $\mathbb R$. Spróbujmy teraz przeformułować tę zasadę. Mówi ona, że istnieją dowolnie duże liczby całkowite, tzn. takie, że ich wartość bezwględna jest dowolnie duża.

Formalniej, zasada Archimedesa jest równoważna ze stwierdzeniem, że $\sup\{|n|:n\in\mathbb Z\} = +\infty$. Fakt ten można jeszcze raz przeformułować: wartość bezwzględna $|\cdot|$ jest niearchimedesowska wtedy i tylko wtedy, gdy $\sup\{|n|:n\in\mathbb Z\} = 1$. Innych możliwości nie ma!

\begin{fkt}Jeśli $\sup\{|n|:n\in\mathbb Z\} = C < +\infty$, to $|\cdot|$ jest niearchimedesowska i $C = 1$.\end{fkt}

\begin{proof}Przyjmijmy nie wprost, że $C > 1$. Skoro tak, to z pewnością istnieje liczba całkowita $m$ taka, że $|m| > 1$. Wiemy jednak, że $|m^k| = |m|^k$ staje się dowolnie duże, kiedy $k$ rośnie, więc $C$ nie może być skończone. Doszliśmy do sprzeczności, więc $C \le 1$, ale $|1| = 1$, więc $C=1$, a sama norma jest niearchimedesowska.\end{proof}

\section{Topologia}
Jeżeli mamy do dyspozycji wartość bezwzględną, to jesteśmy w stanie mierzyć rozmiary, tzn. odległości między liczbymi (ustalić metrykę). Po zdefiniowaniu metryki możemy zdefiniować otwarte i domknięte zbiory oraz badać je, tzn. zajmować się czymś, co nazywa się topologią.

\begin{dff}Niech $\mathbb K$ będzie ciałem, zaś $|\cdot|$ -- wartością bezwzględną na $\mathbb K$. Odległość między dowolnymi dwoma elementami ciała, $x,y\in\mathbb K$, wyraża się wzorem $d(x,y) = |x-y|$ i nazywamy ją metryką indukowaną przez normę.\end{dff}

Wiele własności odległości między liczbami rzeczywistymi może zostać uogólnionych. I tak, dla każdej metryki prawdziwe są następujące stwierdzenia:
\begin{enumx}
\item $(x,y \in \mathbb K) (d(x,x) \ge 0 \wedge d(x,y) = 0 \iff x = y)$
\item $(x,y \in \mathbb K) (d(x,y) = d(y,x))$
\item $(x,y,z \in \mathbb K) (d(x,y) + d(y,z) \ge d(x,z)$
\end{enumx}

Ostatnia nierówność zwana jest nierównością trójkąta, bo wyraża fakt, że suma długości dwóch boków dowolnego trójkąta jest niemniejsza niż długość trzeciego boku (odcinek jest najkrótszą krzywą łączącą dwa punkty). Zbiór, na którym zdefiniowana jest metryka, nazywamy przestrzenią metryczną. Każde ciało z wartością bezwzględną może zostać uczynione taką przestrzenią (z metryką indukowaną przez normę). Chcesz wiedzieć więcej na temat przestrzeni metrycznych -- zapraszam na wykład z topologii.

\begin{prb}$\mathbb K$ jest ciałem topologicznym dzięki metryce $d(x,y)$, bo dodawanie, mnożenie i odwracanie są funkcjami ciągłymi, co zapisujemy:
\begin{enumx}
\item $(\forall x_0, y_0 \in \mathbb K) (\forall \varepsilon > 0) (\exists \delta > 0)(d(x,x_0) < \delta \wedge d(y,y_0) < \delta \implies d(x+y,x_0+y_0) < \varepsilon)$
\item $(\forall x_0, y_0 \in \mathbb K) (\forall \varepsilon > 0) (\exists \delta > 0)(d(x,x_0) < \delta \wedge d(y,y_0) < \delta \implies d(xy,x_0y_0) < \varepsilon)$
\item $(\forall x_0 \in \mathbb K\setminus\{0\}) (\forall \varepsilon > 0) (\exists \delta > 0)(d(x,x_0) < \delta \implies x \neq 0 \wedge d(1/x, 1/x_0) < \varepsilon)$
\end{enumx}\end{prb}

\begin{fkt}Niech $|\cdot|$ będzie normą na ciele $\mathbb K$, zdefiniujmy metrykę przez $d(x,y) = |x-y|$. Norma $|\cdot|$ jest niearchimedesowska wtedy i tylko wtedy, gdy dla dowolnych $x,y,z\in\mathbb K$ mamy $d(x,y) \le \max(d(x,z), d(z,y))$.\end{fkt}

\begin{proof}Najpierw implikacja w prawo: Stosujemy własność niearchimedesowską do równania $(x-y) = (x-z) + (z-y)$. Potem implikacja w lewo: Ustalamy $y = -y'$ i $z=0$ w nierówności z faktu: $|x- (-y')| = |x+y'| \le \max (|x-0|, |0-y'|) = \max (|x|, |y'|)$.\end{proof}

Nierówność z ostatniego faktu znana jest jako \emph{nierówność ultrametryczna}. Metrykę, dla której jest prawdziwa, nazywamy ultrametryką. Przestrzeń z ultrametryką nazywamy przestrzenią ultrametryczną. Okazuje się, że takie przestrzenie zachowują się dziwnie, co nie powinno dziwić, bo sposób mierzenia odległości jest dziwny. To ciekawe, że znalazły już swoje zastosowanie: używa się ich do opisu szkła spinowego.

\begin{fkt}Niech $\mathbb K$ będzie ciałem, zaś $|\cdot|$ niearchimedesowską normą bezwzględną na $\mathbb K$. Jeśli $x,y$ są ustalonymi elementami $\mathbb K$ o różnych normach, tzn. $|x| \neq |y|$, to $|x+y| = \max(|x|, |y|)$.\end{fkt}

\begin{proof}Bez straty ogólności zakładamy, że $|x| > |y|$, jeśli jest inaczej, zamieniamy $x$ i $y$ miejscami. Wiemy, że $|x+y| \le |x| = \max(|x|, |y|)$. Z drugiej strony, $x = (x+y) - y$, więc $|x| \le \max(|x+y|, |y|)$. Pamiętajmy jednak, że $|x| > |y|$, więc musi być $\max(|x+y|, |y|) = |x+y|$. Pokazaliśmy, że $|x+y| \le |x|$ i $|x+y| \ge x$, a zatem $|x| = |x+y|$.\end{proof}

Efektem ubocznym jest inny interesujący fakt, którego znaczenie kiedyś docenimy:

\begin{fkt}Wszystkie \emph{trójkąty} w ultraprzestrzeni są równoramienne.\end{fkt}

\begin{proof}Niech $x,y,z\in\mathbb K$ będą elementami naszej przestrzeni, wierzchołkami \emph{trójkąta}. Długości jego boków wyrażają się przez $d(x,y) = |x-y|$, $d(y,z) = |y-z|$, $d(x,z) = |x-z|$. W oczywisty sposób zachodzi $(x-y) + (y-z) = (x-z)$, więc wzywamy do pomocy poprzedni fakt: jeśli $|x-y| \neq |y-z|$, to $|x-z|$ jest równe większej z dwóch pozostałych liczb, co dowodzi równoramienności \emph{trójkąta}.  \end{proof}

To, co udało nam się pokazać, jest bardzo nieintuicyjne, a jednocześnie ma duży wpływ na topologię naszego ciała. Może spróbujemy zobaczyć co stanie się, kiedy ograniczymy się do $p$-adycznej wartości bezwzględnej? Nie zaszkodzi.

Jak zwykle, piszemy $|x| = p^{-v_p(x)}$. Dla klarowności ograniczymy się do przypadku $x,y\in\mathbb Z$. Umówmy się, że $v_p(x) = a$ oraz $v_p(y)= b$, czyli $x = p^a x'$, $y = p^m y'$ oraz $p\nmid x'y'$. Przetłumaczmy to na wartości bezwzględne: $|x| = p^{-a}$ i $|y| = p^{-b}$. 

Okazuje się, że $|x| > |y|$, jeśli tylko $a < b$, niech zatem $b = a + \varepsilon$ (oczywiście $\varepsilon > 0$). Skoro tak, to $x  +y = p^ax' + p^{a+\varepsilon} y' = p ^a\left(x' + p^\varepsilon y'\right)$,
przy czym $p\nmid x'$, więc $p \nmid (x' + p^\varepsilon y')$ i $v_p(x+y) = a$, a stąd wiemy już, że $|x+y | = p^{-a} = |x|$, tak jak przypuszczaliśmy.

Zajmijmy się teraz drugim przypadkiem, tj. niech $|x| = |y|$ i $a = b$.  Teraz $x+y = p^a(x'+y')$. Chociaż $p\nmid x'$ i $p \nmid y'$, to może się zdarzyć, że $p | (x'+y')$. Jedyne, czego możemy być pewni, to to, że $v_p(x+y) \ge n = \min(v_p(x), v_p(y))$. To oznacza jednak, że $|x+y| \le \max(|x|, |y|) = |x| = |y|$, mogliśmy spodziewać się oczekiwanego.

\begin{prb}Rozpatrzmy $5$-adyczną topologię na $\mathbb Q$. Jakie są długości boków trójkąta o wierzchołkach $2/15$, $1/5$, $7/15$?\end{prb}

W każdej przestrzeni metrycznej od trójkątów dużo bardziej ciekawsze są kule. Chyba nikt nie przypuszcza, że będą zachowywać się normalnie. Zacznijmy od podania ich definicji.

\begin{dff}Niech $\mathbb K$ będzie ciałem z wartością bezwględną  $|\cdot|$, $a$ elementem $\mathbb K$, zaś $r$ dodatnią liczbą rzeczywistą. Kula otwarta to zbiór $B(a,r)$, kula domknięta to zbiór $\overline{B}(a,r)$.
\begin{align*}
B(a,r) & = \left\{x\in\mathbb K: d(x,a) < r \right\} = \left\{x\in\mathbb K: |x-a| < r \right\} \\
\overline{B}(a,r) & = \left\{x\in\mathbb K: d(x,a) \le r \right\} = \left\{x\in\mathbb K: |x-a| \le r \right\}
\end{align*}
\end{dff}

Garść przydatnych definicji: {Zbiór jest otwarty, jeżeli dowolny jego element należy do otwartej kuli zawartej w tym zbiorze. Zbiór jest domknięty, jeżeli jego dopełnienie jest zbiorem otwartym. Punkt nazywamy brzegowym, jeżeli dowolna kula o środku w tym punkcie zawiera punkty należące i nienależące do zbioru. Zbiór jest domknięty wtedy i tylko wtedy, gdy zawiera wszystkie swe punkty brzegowe.}

To są standardowe definicje kul dla dowolnej przestrzeni metrycznej. Kule otwarte są pierwowzorem dla zbiorów otwartych, kule domknięte -- dla zbiorów domknietych.

Powyższe stwierdzenie jest prawdziwe niezależnie od rodzaju wartości bezwzględnej. Poniższe jest raczej zaskakujące.

\begin{fkt}Niech $\mathbb K$ będzie ciałem z niearchimedesowską wartością bezwzględną.

\begin{enumx}
\item Każdy punkt zawarty w kuli otwartej jest jej środkiem: $b\in B(a,r) \Rightarrow B(a,r) = B(b,r)$
\item Każdy punkt zawarty w kuli zamkniętej jest jej środkiem: $b\in \overline{B}(a,r) \Rightarrow \overline{B}(a,r) = \overline{B}(b,r)$
\item Zbiór $B(a,r)$ jest otwarty i domknięty.
\item Zbiór $\overline{B}(a,r)$ jest otwarty i domknięty dla $r\neq 0$.
\item Dowolne dwie kule otwarte są albo rozłączne albo zawarte jedna w drugiej. Niech $a,b\in\mathbb K$, $r,s\in\mathbb R_+^\times$. Wtedy $B(a,r) \cap B(b,s) \neq \emptyset$ wtedy i tylko wtedy, gdy $B(a,r) \subset B(b,s)$ lub $B(a,r) \supset B(b,s)$.
\item Dowolne dwie kule zamknięte są albo rozłączne albo zawarte jedna w drugiej. Niech $a,b\in\mathbb K$, $r,s\in\mathbb R_+^\times$. Wtedy $\overline{B}(a,r) \cap \overline{B}(b,s) \neq \emptyset$ jest równoważne z $\overline{B}(a,r) \subset \overline{B}(b,s)$ lub $\overline{B}(a,r) \supset \overline{B}(b,s)$.
\end{enumx}\end{fkt}

\begin{proof}
Dziwność tych stwierdzeń opiera się na fakcie: wszystkie trójkąty są równoramienne.

\textbf{1.} $b\in B(a,r)$ wtedy i tylko wtedy, gdy $|b-a| < r$. Biorąc dowolne $x$ takie, że $|x-a| < r$, korzystamy z własności niearchimedesowskiej by napisać, że zachodzi  $|x-b| \le \max (|x-a|, |b-a|) < r$, czyli $x\in B(b,r)$. Wiemy, że $B(a,r) \subset B(b,r)$, ale możemy pokazać zawieranie w drugą stronę i dostać równość zbiorów.

\textbf{2.} Zamiana $<$ na $\le$ w dowodzie własności 1.

\textbf{3.} Otwarta kula $B(a,r)$ jest zawsze zbiorem otwartym (jednolinijkowy dowód: dowolny $x$ w $B(a,r)$ jest w $B(a,r)$, która zawiera się w $B(a,r)$!). Weźmy dowolny $x$ na brzegu $B(a,r)$, to oznacza, że kula o środku w $x$ musi zawierać punkty, które należą do $B(a,r)$. Weźmy $s\le r$ i spójrzmy na otwartą kulę $B(x,s)$. Ponieważ $x$ jest punktem brzegowym, to $B(a,r) \cap B(s,x) \neq \emptyset$, więc istnieje $y\in B(a,r) \cap B(s,x)$. To oznacza, że $|y-a| < r$ i $|y-x| < s \le r$. Aplikując nierówność niearchimedesowską dostajemy $|x-a| \le \max(|x-y|, |y-a|) < \max(s,r) \le r$, czyli $x\in B(a,r)$. Każdy punkt brzegowy $B(a,r)$ należy do $B(a,r)$, czyli zbiór jest domknięty.

\textbf{4.} Analogicznie do 3.

\textbf{5.} Bez straty ogólności możemy rozpatrzeć tylko przypadek, gdy $r\le s$. Jeżeli iloczyn jest niepusty, to istnieje $c \in B(a,r) \cap B(b,s)$. Z pierwszej własności wnioskujemy, że $B(a,r) = B(c,r)$ i $B(b,s) = B(c,s)$. Zatem: $B(a,r) = B(c,r) \subset B(c,s) = B(b,s).$

\textbf{6.} Analogicznie do 5. z użyciem 2. zamiast 1.\end{proof}
\begin{prb}Dlaczego musimy założyć w punkcie 4., że $r\neq 0$?\end{prb}

\begin{prbh}Niech $\mathbb K = \mathbb Q$ i $|\cdot| = |\cdot|_p$. Domknięta kula $\overline{B}(0,1)$jest sumą rozłącznych kul otwartych\footnote{To jest inny dowód faktu, że jednostkowa kula domknięta jest otwarta, gdyż suma zbiorów otwartych jest otwarta.}: $\overline{B}(0,1) = \bigcup_{k=0}^{p-1} B(k,1)$. \end{prbh}

\begin{prb}Dlaczego $B(1,1) = B(1,1/2) = \overline{B}(1,1/5)$ z $5$-adyczną normą na $\mathbb Q$? Co tu się dzieje?\end{prb}

Zbiory spójne nie wow, przestrzeń taka rozłączna.

\section{Algebra}
Nie możemy skupiać naszej uwagi wyłącznie na geometrycznym aspekcie normy. Jeżeli nie umiemy algebry, to jedyne, co nam pozostaje, to wiara w to, że kiedyś nadejdzie taki dzień, że zostanie przez nas zrozumiana. Okazuje się na przykład, że połączenia między (niearchimedesowskimi) wartościami bezwzględnymi i algebraiczną strukturą ciała są dosyć poważne, więc czemu nie mielibyśmy się temu przyjrzeć?

Chcemy rozszerzyć naszą teorię i podać jeszcze jeden argument przemawiający za słusznością analogii Hensela (tj. podobieństwa między $\mathbb Q$ i ciałem funkcji wymiernych). Niech zatem $F$ będzie ciałem (skończonym lub nie), $F[t]$ będzie pierścieniem wielomianów o współczynnikach z $F$, zaś $F(t)$ będzie ciałem funkcji wymiernych nad $F$. Zdefiniujmy kilka waluacji.

Przede wszystkim, dla dowolnego wielomianu $\phi(t) \in F[t]$

\section{Liczby $p$-adyczne}
Opanowaliśmy już wystarczająco dużo materiału, by móc ograniczyć nasze ogólne rozważania do specjalnego przypadku, ciała $\mathbb Q$ liczb wymiernych. Znamy już kilka przykładów norm: klasyczną, trywialną i $p$-adyczną wartość bezwzględną. Być może niektórzy zastanawiają się, czy to już wszystkie możliwości. Zanim odpowiemy na to ważne pytanie, będziemy musieli zdefiniować co to znaczy, że dwie normy są \emph{takie same}.

\begin{dff}Dwie normy $|\cdot|_1$ i $|\cdot|_2$ na ciele $\mathbb K$ nazywamy równoważnymi, jeśli definiuję tę samą topologię na $\mathbb K$. Innymi słowy, zbiór jest otwarty względem jednej z norm wtedy i tylko wtedy, gdy jest otwarty względem drugiej.\end{dff}

Łatwiej powiedzieć aniżeli sprawdzić! Lepiej znajdźmy jakieś kryterium, które jest bardziej używalne.

\begin{fkt}Niech $|\cdot|_1$ i $|\cdot|_2$ będą normami na ciele $\mathbb K$. Następujące warunki są równoważne:
\begin{enumx}
\item $|\cdot|_1$ i $|\cdot|_2$ są równoważne,
\item dla każdego $x\in\mathbb K$ mamy $|x|_1 < 1 \iff |x|_2 < 1$,
\item istnieje takie dodatnie $a \in \mathbb R$, że dla każdego $x\in\mathbb K$ mamy $|x|_1 = |x|_2^a$.
\end{enumx}\end{fkt}
\begin{proof}Udowodnimy cykl implikacji: $1\implies2\implies3\implies1$.

\textbf{Pierwsza}. Każdy ciąg zbieżny odnośnie jednej z wartości bezwzględnych musi być zbieżny odnośnie drugiej (bo topologie są równoważne). Łatwo jednak widać, że $\lim_{n\rightarrow \infty} x^n = 0$ odnośnie topologii indukowanej przez wartość bezwzględną $|\cdot|$ wtedy i tylko wtedy, gdy $|x|<1$.

\textbf{Druga}. To najciekawsza część dowodu. Najpierw będziemy musieli znaleźć $a$: ustalimy dowolne $x_0$ i powiemy, że $a$ jest jedyną liczbą, dla której $|x_0|_1 = |x_0|_2^a$. Pokażemy też, że to $a$ jest dobre dla każdego $x_0 \in \mathbb K$, a nie jedynie dla $x_0$. Ustaliliśmy już plan, działajmy!

Weźmy dowolne $x_0\in\mathbb K\setminus\{0\}$ takie, że $|x_0|_1 < 1$\footnote{Dlaczego jest ważne, by $|x_0|_1 < 1$? Co się stanie, jeżeli takiego $x_0$ nie ma?}. Na mocy założenia wiemy, że $|x_0|_2 < 1$, więc istnieje takie rzeczywiste $a > 0$, że $|x_0|_1 = |x_0|_2^a$. Weźmy teraz dowolne $x\in\mathbb K\setminus\{0\}$. Jeżeli $|x|_1 = |x_0|_1$, to musimy mieć $|x|_2 = |x_0|_2$: w przeciwnym przypadku albo $x/x_0$, albo $x_0/x$ miałoby $|\cdot|_2$ mniejszą niż $1$, wbrew \textbf{2}. W takim razie $|x|_1 = |x|_2^a$ jest prawdą. Jeżeli $|x|_1 = 1$, to (na mocy \textbf{2}. użytego wobec $x$ lub $1/x$) $|x|_2 = 1$, wszystko gra. Zauważmy wreszcie, że równość dla pewnego $x$ implikuje równość dla jego potęg: $|x_0^n|_1 = |x_0^n|_2^a$.

Teraz możemy założyć, że $|x|_k \neq 1$ i $|x|_k \neq |x_0|_k$ dla $k = 1,2$. Tak jak przed chwilunią, ustalmy takie $b$, że $|x|_1 = |x|_2^b$ -- to oznacza, że $|x^n|_1 = |x^n|_2^b$ dla $n\in\mathbb N$. W szczególności możemy założyć, że $|x|_1 < 1$ (jeśli tak nie jest, to zastąpmy $x$ przez $1/x$), co implikuje $|x|_2 < 1$. Chcemy pokazać, że $a = b$. Ustalmy dwie liczby naturalne, $u$ i $t$. Zlogarytmujemy pierwszą i ostatnią nierówność stronami. \[|x|_1^u < |x_0|_1^t \iff \left|\frac{x^u}{x_0^t}\right|_1 < 1 \iff \left|\frac{x^u}{x_0^t}\right|_2 < 1 \iff |x|_2^u < |x_0|_2^t.\]
\[\frac{u}{t} < \frac{\log |x_0|_1}{\log |x|_1} \iff \frac{u}{t} < \frac{\log |x_0|_2}{\log |x|_2}.\]

Zbiór ułamków mniejszych niż pierwszy iloraz i zbiór ułamków mniejszych niż drugi iloraz są tym samym zbiorem. Zbiór liczb wymiernych jest gęsty, więc oba ilorazy muszą być równe! $\star$: ustalmy $|x_0|_1 = |x|_2^a$ i $|x|_1 = |x|_2^b$, wtedy pierwszy iloraz będzie równy $a$, zaś drugi będzie równy $b$.
\[\frac{\log |x_0|_1}{\log |x|_1} = \frac{\log |x_0|_2}{\log |x|_2} \iff \frac{\log |x_0|_1}{\log |x_0|_2} = \frac{\log |x|_1}{\log |x|_2} \overset{\star}{\iff} a = b\]

\textbf{Trzecia}. Na mocy założenia mamy $|x-a|_1 < r \Leftrightarrow |x-a|_2^a < r \Leftrightarrow |x-a|_2 < r^{1/a}$,
czyli każda kula otwarta odnośnie pierwszej wartości bezwzględnej jet otwartą kulą (choć o innym promieniu) odnośnie drugiej wartości bezwzględnej, więc topologie przez nie indukowane są identyczne [dlaczego?], a same wartości bezwzględne są równoważne.\end{proof}

Uff! Nadszedł wreszcie czas na jedno z ważniejszych twierdzeń, z dowodem którego przyjdzie nam się zmierzyć. Mówi ono, że udało nam się znaleźć już wszystkie możliwe wartości bezwzględne na $\mathbb Q$!

\begin{fkt}[Ostrowski, 1916] Każda nietrywialna norma $|\cdot|$ na $\mathbb Q$ jest równoważna z normą $|\cdot|_p$ dla $p \in \mathbb P \cup \{\infty\}$.\end{fkt}

\textbf{Dowód}. Załóżmy najpierw, że $|\cdot|$ jest archimedesowska. Niech $n_0 \in\mathbb N$ będzie najmniejszą liczbą, dla której $|n_0| > 1$. Możemy wskazać teraz $a\in\mathbb R_+$ taką, że $|n_0| = n_0^a$. Zapiszmy teraz dowolną liczbę $n\in\mathbb N$ w pozycyjnym systemie o podstawie $n_0$: \[n = a_0 + a_1 n_0 + \cdots + a_s n_0^s, \text{ gdzie }0 \le a_i < n_0 \text{ i } a_s \neq 0, \text{ zaś } s = \left\lfloor\frac{\log n}{\log n_0}\right\rfloor, \text{ bo }n_0^s \le n < n_0^{s+1}.\]
Ze względu na wybór $n_0$ mamy $a_i < n$, czyli $|a_1| \le 1 < |n| $. To wystarcza do oszacowania $|n|$ z góry:
\begin{align*}
|n| = 
\left|\sum_{k=0}^s a_k n_0^{k}\right| = 
\sum_{k=0}^s \left|a_k n_0^{k}\right| = 
\sum_{k=0}^s \left|a_k\right| n_0^{ka} \le
\sum_{k=0}^s n_0^{ka} = 
n_0^{sa} \sum_{k=0}^s n_0^{-ka} \le
n^a \sum_{k=0}^\infty n_0^{-ka} = n^a \cdot \frac{n_0^a}{n_0^a - 1}\end{align*}
Nieskończona suma jest skończoną stałą, $C$. Pokazaliśmy, że $|n| \le Cn_0^{sa} \le Cn^a$, czas na nieczyste zagranie. Nierówność jest prawdziwa zawsze, więc możemy ustalić dowolnie duże $m$ i zamienić $n$ na $n^m$! Mamy teraz $|n|^m \le Cn^{am}$ (możemy tak napisać, bo $C$ nie zależy od $n$!), pierwiastkując nierówność stronami i przechodząc do granicy ($\lim_{m\to\infty} \sqrt[m]{C} = 1$) pozbędziemy się stałej. A zatem wiemy już, że $| n| \le n^a$.

Skorzystamy jeszcze raz z rozwinięcia $n$, by odwrócić tę nierówność.
\begin{align*}
|n| & \overset{\star_1}{\ge}
\left|n_0^{s+1}\right| - \left|n_0^{s+1} - n\right| \overset{\star_2}{\ge}
n_0^{a(s+1)} - \left(n_0^{s+1} - n\right)^a \\ & \overset{\star_3}{\ge} 
n_0^{a(s+1)} - \left(n_0^{s+1} - n_0^s\right)^a  \\ & \overset{\star_4}{=}
n_0^{a(s+1)} \left[1 - \left( 1 - \frac{1}{n_0}\right)^a\right]  \overset{\star_5}{=}
C' n_0^{a(s+1)} \overset{\star_6}{\ge}
C' n^a 
\end{align*}
Przekształcenia, których użyliśmy, nie są takie oczywiste, więc przyjrzyjmy się im uważniej.
\begin{itemx}
\item $\star_1$: wiemy, że $|n_0^{s+1}| = |n+n_0^{s+1}-n| \le |n| + |n_0^{s+1} - n|$, drugi składnik przenosimy na lewą stronę.
\item $\star_2$: korzystamy z faktu: $|n_0| = n_0^a$, odjemnik szacujemy z góry przy użyciu $| n| \le n^a$.
\item $\star_3$: z naszego rozwinięcia wynika, że $n \ge n_0^s$
\item $\star_4$: po prostu wyłączamy czynnik przed nawiasy.
\item $\star_5$: liczba w nawiasach zależy od $n_0$ i $a$, ale nie od $n$, więc jest stałą!
\item $\star_6$: ponownie korzystamy z rozwinięcia, by przekonać się, że $n_0^{s+1} > n$.
\end{itemx}

Korzystamy z tej samej sztuczki, by pozbyć się stałej $C'$. Połączmy oba oszacowania: $|n| = n^a$. Norma jest multiplkatywna, więc $|x| = |x|_\infty^a$ dla każdego $x\in\mathbb Q$ (norma równoważna klasycznej wartości bezwzględnej)

Rozpatrzmy teraz drugi przypadek, gdy $|\cdot|$ jest niearchimedesowska. Niech $n_0$ będzie najmniejszą liczbą, która spełnia $|n| \le 1$ (norma ma być nietrywialna!). Okazuje się, że $n_0$ musi być liczbą pierwszą -- jeśli bowiem $n_0 = n_1 \cdot n_2$ i $1 < n_1, n_2 < n_0$, to $|n_1| = |n_2| = 1$, czyli $1 > |n_0| = |n_1| \cdot |n_2| = 1$, sprzeczność. Niech $p = n_0$. Twierdzimy, że $|n| = 1$, jeśli $n$ nie jest podzielne przez $p$. Jeżeli $p\nmid n$, to $n = pr+s$, gdzie $0 < s < p$. Wiemy, że $|s| = 1$ (ze względu na wybór $p$). Norma jest niearchimedesowska, więc $|r| \le 1$, czyli $|pr| < 1$. Wszystkie trójkąty są równoramienne, więc $|n| = 1$. Wreszcie, dla dowolnego $n\in\mathbb Z$ możemy napisać, że $n = p^vn_0$, gdzie $p\nmid n_0$. Korzystając z definicji normy: $|n| = |p|^v |n_0| = |p|^v = (1/|p|)^{-v}$, a to oznacza, że norma jest równoważna z $p$-adyczną, bo $|p| < 1$.

\begin{prbh}W naszym dowodzie jest pewna luka -- gdybyśmy wiedzieli, że $n_0 = 2$, to wszystko byłoby w porządku. Niestety, $n_0$ może okazać się być czymś dużo większym. Jeżeli $n < n_0$, to $s = 0$, czyli rozwinięciem $n$ w pozycyjnym systemie o podstawie $n_0$ to po prostu $n$. Co teraz?\end{prbh}
\textbf{Product Formula}. \textbf{$Q$ nie jest zupełne}. \textbf{Ciągi Cauchy'ego}.
\end{multicols}
{}
%\end{multicols}

% PROBLEM 43


\newpage
%\begin{center}\textbf{\Huge Wskazówki i rozwiązania problemów}\end{center}
\section*{Wskazówki i rozwiązania}
%\begin{multicols}{2}

\begin{sol}Kolejno: $3$, $7$, $0$, $4$, $2$, $0$, $0$, $1$.\end{sol}
\begin{sol}Sugestywna wskazówka: jeśli $\alpha$ to potęga, w jakiej wchodzi $p$ w rozkładzie na czynniki pierwsze $n!$, to\[\alpha = \sum_{i=1}^{\infty} \left[\frac{n}{p^i}\right]\]\end{sol}
\begin{sol}Patrz poprzednie ćwiczenie.\end{sol}
\begin{sol}Co możesz powiedzieć o waluacji $v_p(x)$ dla dowolnego $p\in\mathbb P$?\end{sol}
\begin{sol}Wystarczy policzyć waluacje i skorzystać z definicji: $1/7$, $1/7$, $1$, $343$.\end{sol}
\begin{sol}Zero!\end{sol}
\begin{sol}Ustalenie $c=p$ pozwala zapisać Product Formula w ładniejszej postaci.\end{sol}

\begin{sol}Dodawanie -- potrzebna będzie nierówność trójkąta:
$d(x+y, x_0 + y_0) = |(x+y) -(x_0+y_0) | =\\=  |(x-x_0)+ (y-y_0)| \le |x-x_0| + |y-y_0|$
Mnożenie -- odrobina spostrzegawczości może być przydatna, bowiem $xy - x_0y_0 = x(y-y_0) + y_0(x-x_0)$. Wreszcie odwracanie, tutaj można użyć tożsamości $1/x - 1/{x_0} = {(x_0-x)}/{xx_0}$.\end{sol}

\begin{sol}$1$, $5$, $5$.  \end{sol}

\begin{sol}Dla $r = 0$ nasza kula degeneruje się do punktu, który nie jest zbiorem otwartym (chyba, że wartość bezwzględna była trywialna). Dla porównania, otwarte kule o zerowym promeniu są zbiorami pustymi, a zatem są otwarte i domknięte niezależnie od okoliczności.\end{sol}

\begin{sol}Trzeba sprawdzić rozłączność i domkniętość!\end{sol}
\begin{sol}Wskazówka: $1/5 < 1 < 5$.\end{sol}

\begin{sol}Popatrz co się dzieje dla $n=1$, $1<n<n_0$, a następnie zauważ, że $n^m$ może stać się dowolnie duże.\end{sol}



%\begin{center}\textbf{\Huge Alternatywne dowody}\end{center}
\section*{Alternatywne dowody}
\textbf{Dowód tw. Ostrowskiego, część druga}. Rozpatrzmy teraz drugi przypadek, gdy $|\cdot|$ jest niearchimedesowska. Niech $n_0$ będzie najmniejszą liczbą, która spełnia $|n| \le 1$ (norma ma być nietrywialna!). Okazuje się, że $n_0$ musi być liczbą pierwszą -- jeśli bowiem $n_0 = n_1 \cdot n_2$ i $1 < n_1, n_2 < n_0$, to $|n_1| = |n_2| = 1$, czyli $1 > |n_0| = |n_1| \cdot |n_2| = 1$, sprzeczność. Niech $p = n_0$. Twierdzimy, że $|q| = 1$, jeśli $q$ jest liczbą pierwszą różną od $p$. Najłatwiej dowodzić tego nie wprost: jeśli $|q| < 1$, to dla pewnego dużego $N$ mamy $|q^N| = |q|^N \le 1/2$, dla pewnego dużego $M$ mamy $|p^M| < 1/2$. Możemy znaleźć dwie liczby całkowite $a, b$ takie, że $bp^M + aq^N = 1$, gdyż $p^M$ i $q^N$ są względnie pierwsze. Ale
\[1 =|1| \le |bp^M| + |aq^N| = |b| \cdot |p^M| + |a| \cdot |q^N|\le |p^M| + |q^N| < \frac{1}{2} + \frac{1}{2} = 1\]
To już prawie koniec dowodu: każda liczba całkowita $n$ może zostać zapisana jako iloczyn liczb pierwszych $n = p_1^{\alpha_1} \cdot p_2^{\alpha_2} \cdot \ldots \cdot p_r^{\alpha_r} $. Z tego wynika, że $|a| = |p_1|^{\alpha_1} \cdot |p_2|^{\alpha_2} \cdot \ldots \cdot |p_r|^{\alpha_r}$. Jedyne $|p_i|$ różne od jedynki to $|p|$, wtedy $\alpha_i = v_p(n)$ i $|n| = |p|^{v_p(n)}$. Korzystamy z multiplikatywności, by rozszerzyć rozumowanie na $\mathbb Q$: ta norma jest równoważna z $|\cdot|_p$.
\newpage
%*\end{multicols}

\end{document}

09 / 01 / 2014: Wartość bezwzględna, waluacja.
16 / 01 / 2014: $p$-adyczna wartość bezwzględna